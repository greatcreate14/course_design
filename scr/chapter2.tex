\section{上部结构}
\subsection{上部结构资料}

拟建建筑物为六层钢筋混凝土结构,长30m,宽9.6m。室外地坪标高同自然地面,室内外高差450mm。柱截面尺寸均为400mm×400mm,横向承重,柱网布置如图
\ref{fig:柱网布置图}所示。 
\\
\begin{figure}[htbp] % h:here, t:top, b:bottom, p:page
  \centering
  % 这里直接插入生成的 pdf 文件
  % width=0.9\textwidth 表示图片宽度占文本宽度的 90%,自动缩放
  \includegraphics[width=1\textwidth]{柱网布置图.pdf} 
  
  \caption{柱网布置图} % 图标题
  \label{fig:柱网布置图} % 用于引用的标签
\end{figure}


\subsection{上部结构作用}

上部结构作用在柱底的荷载效应标准组合值如表\ref{tab:柱底荷载效应标准组合值}所示,该表中弯矩$M_k$ 、
水平力$V_k$均为横向方向。上部结构作用在柱底的荷载效应基本组合值如表\ref{tab:柱底荷载效应基本组合值}所示,
该表中弯矩$M$、水平力$V$均为横向方向。

\begin{table}[htbp]
  \centering
  \caption{柱底荷载效应标准组合值}
  \label{tab:柱底荷载效应标准组合值}
  \begin{tabular}{ccc}
      \toprule
      $F_k \, (\mathrm{kN})$ & $M_k\, (\mathrm{kN} \cdot \mathrm{m})$ & $V_k \, (\mathrm{kN})$ \\
      \midrule
      3728 & 277 & 200 \\
      \bottomrule
  \end{tabular}
\end{table}

\begin{table}[htbp]
  \centering
  \caption{柱底荷载效应基本组合值}
  \label{tab:柱底荷载效应基本组合值}
  \begin{tabular}{ccc}
      \toprule
      $F \, (\mathrm{kN})$ & $M\, (\mathrm{kN} \cdot \mathrm{m})$ & $V \, (\mathrm{kN})$ \\
      \midrule
      4641 & 331 & 235 \\
      \bottomrule
  \end{tabular}
\end{table}

\subsection{材料}
混凝土强度等级为C30,钢筋采用HRB400级。