\section{桩基设计}

\subsection{桩基持力层和桩长}

根据《建筑桩基技术规范》(JGJ 94-2008) 第 3.3.3 条规定:应选择较硬土层作为桩端持力层。桩端全断面进入持力层的深度,对于黏性土、粉土不宜小于 $2d$,砂土不宜小于 $1.5d$,碎石类土不宜小于 $1d$。

本设计选用第\circled{5}层粉质粘土混卵砾石作为桩端持力层。
桩截面边长 $d = 500\text{mm}$。
设计桩端进入持力层深度为:
\begin{equation*}
    3.0d =  3 \times 0.5=1.5 \text{m}
\end{equation*}

设地面标高为-0.45m,初步设计承台高1.0m,承台地面埋置深度标高-1.45m,根据《建筑桩基技术规范》(JGJ 94-2008) 第 4.2.4 条规定:桩嵌入承台内的长度对中等直径桩不宜小于50mm;对大直径桩不宜小于100mm。本设计桩顶伸入承台100mm。


由表\ref{tab:地基岩土物理力学参数}场地地质条件可知各个土层的厚度,故桩基有效桩长$l$为:
\begin{equation*}
    l = 2+3.5+5+10+1.5-1 =21 \text{m}
\end{equation*}

\subsection{单桩竖向承载力的确定}

由表\ref{tab:桩的极限指标}可知桩的极限侧阻力标准值 $q_{sk}$ 和极限端阻力标准值 $q_{pk}$
根据规范经验公式,其单桩竖向承载力计算公式为:
\begin{equation*}
    Q_{uk}=Q_{sk}+Q_{pk}=u\Sigma q_{sik} l_i + q_{pk} A_p
\end{equation*}
桩身周长:
\begin{equation*}
    u=4\times 0.5 = 2 \text{m}
\end{equation*}
桩截面积:
\begin{equation*}
    A_p = 0.5 \times 0.5 = 0.25 \text{m}^2
\end{equation*}
则:
\begin{equation*}
    \begin{aligned}
        Q_{sk}  &= u\Sigma q_{sik} l_i = 2.0 \times (27 \times 5 + 38 \times 10 + 43 \times 1.5) = 1159 \text{ kN}\\
    Q_{pk} &= q_{pk} A_p = 1800 \times 0.25 = 450 \text{ kN} \\
    Q_{uk} &= Q_{sk} + Q_{pk} = 1159 + 450 = 1609 \text{ kN}
    \end{aligned}
\end{equation*}
\\
不考虑群桩效应,估算单桩竖向承载力设计值 $R_a$ 为:
\begin{equation*}
    R_a = \frac{Q_{uk}}{K} = \frac{1609}{2} =804.5 \text{ kN}
\end{equation*}

\subsection{确定桩数}  
根据《建筑桩基技术规范》(JGJ 94-2008) 第 5.1.1 条,单桩竖向力计算需考虑承台及覆土自重 $G_k$。因承台尺寸尚未确定,暂按上部结构荷载的 15\% 估算 $G_k$(即 $G_k = 0.15 F_k$),则初估桩数公式为:
$$n \ge \frac{1.15 F_k}{R_a} $$

则:
$$n \ge \frac{1.15 F_k}{R_a} = \frac{1.15 \times 3728}{804.5} = 5.3$$
桩取6根。