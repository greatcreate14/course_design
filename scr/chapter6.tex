\section{单桩承载力验算}

\subsection{单桩竖向承载力验算}

根据《建筑桩基技术规范》(JGJ 94-2008) ,当按单桩承载力特征值进行计算时,荷载效应取其效应的标准组合值。由于桩基所处场地内无可液化沙土,粉土问题,因此可不进行地震效应的竖向承载力验算。
\\
承台及其上填土的总重为:
$$
    G_k = 20 \times 4.5 \times 2.7 \times 1.7 = 413.1 \text{ kN}
$$

计算时取表\ref{tab:柱底荷载效应标准组合值}荷载的标准组合\\
则基桩平均竖向力 $Q_k$:
$$
    Q_k = \frac{F_k + G_k}{n} = \frac{3728 + 413.1}{6} = 690.18 \text{ kN}
$$
基桩最大竖向力 $Q_{kmax}$:
\begin{equation*} 
    \begin{aligned} 
        Q_{kmax} &= \frac{F_k + G_k}{n} + \frac{(M_k + V_k h) y_{max}}{\sum y_i^2} \\ 
        &= 690.18 + \frac{(277 + 200 \times 1.5) \times 2.0}{16} \\ 
        &= 770.32 \text{ kN} 
    \end{aligned} 
\end{equation*}
基桩最小竖向力 $Q_{kmin}$:
\begin{equation*} 
    \begin{aligned} 
        Q_{kmin} &= \frac{F_k + G_k}{n} - \frac{(M_k + V_k h) y_{max}}{\sum y_i^2} \\ 
        &= 690.18 - \frac{(277 + 200 \times 1.5) \times 2.0}{16} \\ 
        &= 610.04 \text{ kN} 
    \end{aligned} 
\end{equation*}

因此
$$
Q_k<R_a
$$
$$
Q_{kmax}< 1.2R_a 
$$
$$
Q_{kmin}> 0
$$

满足设计要求,故设计是合理的。

\subsection{单桩水平承载力验算}
根据表\ref{tab:柱底荷载效应标准组合值}荷载的标准组合水平力$V_k$计算
\\
单桩水平力:
$$
H_{ik} = \frac{V_k}{n} = \frac{200}{6} = 33.33 \text{ kN}
$$

