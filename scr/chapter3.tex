\section{桩基础设计}

\subsection{确定承台尺寸}
桩数为6根,选用矩形承台,桩排布为2行3列。


根据《建筑桩基技术规范》(JGJ 94-2008) 第 3.3.3 条的表 3.3.3-1 、第 4.2.1 条、第 4.2.3 条和第 4.2.4 条规定: 


独立柱下桩基承台的最小宽度不应小于500mm,边桩中心至承台边缘的距离不应小于桩的直径或边长,且桩的外边缘至承台边缘的距离不应小于150mm。对于墙下条形承台梁,桩的外边缘至承台梁边缘的距离不应小于75mm。承台的最小厚度不应小于300mm。 


承台底面钢筋的混凝土保护层厚度,当有混凝土垫层时,不应小于 50mm,无垫层时不应小于70mm;此外尚不应小于桩头嵌入承台内的长度。 


桩嵌入承台内的长度对中等直径桩不宜小于50mm;对大直径桩不宜小于100mm。
\\
本设计桩中心距:$$s = 4d = 4 \times 0.45 = 1.8 \text{ m}$$
边桩中心至承台边缘的距离取0.45m,承台底面钢筋的混凝土保护层厚度取70mm,桩顶嵌入承台长度取100mm,设承台高$h = 1.5 \text{m}$,
\\则承台的长边长:$$ a = 2 \times 0.45 + 2 \times 1.8 = 4.5 \text{ m}$$
承台短边长: $$
    b  = 2 \times 0.45 + 1.8 = 2.7 \text{ m}
$$
承台有效高度: $$
    h_0 = 1.5-0.07 = 1.43 \text{ m}
$$

\subsection{单桩竖向承载力验算}

根据《建筑桩基技术规范》(JGJ 94-2008) ,当按单桩承载力特征值进行计算时,荷载效应取其效应的标准组合值。由于桩基所处场地内无可液化沙土,粉土问题,因此可不进行地震效应的竖向承载力验算。
\\
承台及其上填土的总重为:
$$
    G_k = 20 \times 4.5 \times 2.7 \times 1.7 = 413.1 \text{ kN}
$$

计算时取表\ref{tab:柱底荷载效应标准组合值}荷载的标准组合\\
则基桩平均竖向力 $Q_k$:
$$
    Q_k = \frac{F_k + G_k}{n} = \frac{3728 + 413.1}{6} = 690.18 \text{ kN}
$$
基桩最大竖向力 $Q_{kmax}$:
\begin{equation*} 
    \begin{aligned} 
        Q_{kmax} &= \frac{F_k + G_k}{n} + \frac{(M_k + V_k h) y_{max}}{\sum y_i^2} \\ 
        &= 690.18 + \frac{(277 + 200 \times 1.5) \times 2.0}{16} \\ 
        &= 770.32 \text{ kN} 
    \end{aligned} 
\end{equation*}
基桩最小竖向力 $Q_{kmin}$:
\begin{equation*} 
    \begin{aligned} 
        Q_{kmin} &= \frac{F_k + G_k}{n} - \frac{(M_k + V_k h) y_{max}}{\sum y_i^2} \\ 
        &= 690.18 - \frac{(277 + 200 \times 1.5) \times 2.0}{16} \\ 
        &= 610.04 \text{ kN} 
    \end{aligned} 
\end{equation*}

因此
$$
Q_k<R_a
$$
$$
Q_{kmax}< 1.2R_a 
$$
$$
Q_{kmin}> 0
$$

满足设计要求,故设计是合理的。

\subsection{单桩水平承载力验算}
根据表\ref{tab:柱底荷载效应标准组合值}荷载的标准组合水平力$V_k$计算
\\
单桩水平力:
$$
H_{ik} = \frac{V_k}{n} = \frac{200}{6} = 33.33 \text{ kN}
$$

\subsection{承台抗弯计算和配筋设计}

承台内力计算荷载采用荷载效应基本组合设计值
\\
则基桩平均竖向力 $N$ :
\begin{equation*}
    \begin{aligned}
        N = \frac{F}{n} = \frac{4641}{6} = 773.5 \text{ kN}
    \end{aligned}
\end{equation*}

基桩最大竖向力 $N_{max}$ :
\begin{equation*}
\begin{aligned}
N_{max} &= \frac{F}{n} + \frac{(M + V h) y_{max}}{\sum y_i^2} \\
&= 773.5 + \frac{(331 + 235 \times 1.5) \times 1.8}{12.96} \\
&= 773.5 + 94.93 \\
&= 868.43 \text{ kN}
\end{aligned}
\end{equation*}

基桩最小竖向力 $N_{min}$ :
\begin{equation*}
\begin{aligned}
N_{min} &= \frac{F}{n} - \frac{(M + V h) y_{max}}{\sum y_i^2} \\
&= 773.5 - 94.93 \\
&= 678.57 \text{ kN}
\end{aligned}
\end{equation*}
