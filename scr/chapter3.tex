\section{选定桩型}

\subsection{桩基持力层的选取}

根据《建筑桩基技术规范》(JGJ 94-2008) 第 3.3.3 条规定:应选择较硬土层作为桩端持力层。桩端全断面进入持力层的深度,对于黏性土、粉土不宜小于 $2d$,砂土不宜小于 $1.5d$,碎石类土不宜小于 $1d$。

本设计选用第\circled{5}层粉质粘土混卵砾石作为桩端持力层。设计桩端进入持力层深度为:3.5 m 。

\subsection{桩的选型与尺寸}
根据《基础工程》\cite{莫海鸿2014基础工程} 4.2.1节中对混凝土桩的描述:


混凝土预制桩的截面有方、圆等各种形状,普通实心方桩的截面边长一般为300~500mm。


本设计选用混凝土预制方桩,桩截面边长取 $d = 450\text{ mm}$


设地面标高为0m,初步设计承台底面埋置深度标高-1.7m,


由表\ref{tab:地基岩土物理力学参数}场地地质条件可知各个土层的厚度,故桩基有效桩长$l$为:
\begin{equation*}
    l = 2+3.5+5+10+3.5-1.7 = 22.3 \text{ m}
\end{equation*}



