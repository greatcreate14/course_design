\section{设计资料}

\subsection{地形}
拟建建筑场地地势平坦,局部堆有建筑垃圾。

\subsection{工程地质条件}
自上而下土层依次如表\ref{tab:场地土层物理力学指标统计表}所示:
\begin{table}[htbp]
  \centering
  \caption{场地土层物理力学指标统计表}
  \label{tab:场地土层物理力学指标统计表}
  \begin{tabular}{clclc}
      \toprule
      层号 & 土层名称 & 层厚 (m) & \multicolumn{1}{c}{状态描述} & $f_{ak}$ (kPa) \\
      \midrule
      \circled{1} & 杂填土 & 2.0 & 稍湿,松散 & 95 \\
      \circled{2} & 淤泥质土 & 3.5 & 流塑 & 65 \\
      \circled{3} & 粉质粘土 & 5.0 & 稍密 & 130 \\
      \circled{4} & 粉质粘土 & 10.0 & 湿,可塑 & 200 \\
      \circled{5} & 粉质粘土混卵砾石 & 4.2 & - & 220 \\
      \circled{6} & 强风化泥质粉砂岩 & 10.0 & - & 300 \\
      \circled{7} & 中风化泥质粉砂岩 & 未揭穿 & - & 1000 \\
      \bottomrule
  \end{tabular}
\end{table}


\subsection{岩土设计技术参数}
岩土设计参数如表\ref{tab:地基岩土物理力学参数}和表\ref{tab:桩的极限指标}所示:

\begin{table}[htbp]
  \centering
  \caption{地基岩土物理力学参数}
  \label{tab:地基岩土物理力学参数}
  % 定义列格式:居中对齐,部分列宽微调以适应页面
  \begin{tabular}{cccccccc}
      \toprule
      土层 & 土层 & 孔隙比 & 含水量 & 液性 & $\gamma$ & $C(\text{kPa})/$ & 压缩模量 \\
      编号 & 名称 & $e$ & $W(\%)$ & 指数 $I_L$ & $(\text{kN}/\text{m}^3)$ & $\Phi(^\circ)$ & $E_s(\text{MPa})$ \\
      \midrule
      \circled{1}& 杂填土 & — & — & — & 19 & 20/10 & 5.0 \\
      \circled{2} & 淤泥质土 & 1.04 & 62.4 & 1.08 & 17 & 8/5 & 3.8 \\
      \circled{3} & 粉质粘土 & — & — & — & 20 & 38/16 & 5.81 \\
      \circled{4} & 粉质黏土 & — & — & — & 19.7 & 42/17.4 & 8.18 \\
      \circled{5} & \begin{tabular}{@{}c@{}}粉质粘土\\混卵砾石\end{tabular} & — & — & — & 20 & 50/20 & 9.0 \\
      \circled{6} & \begin{tabular}{@{}c@{}}强风化泥\\质粉砂岩\end{tabular} & — & — & — & 20.5 & 20/35 & 15 \\
      \circled{7} & \begin{tabular}{@{}c@{}}中风化泥\\质粉砂岩\end{tabular} & — & — & — & 21.5 & 200/40 & — \\
      \bottomrule
  \end{tabular}
\end{table}
\newpage

\begin{table}[htbp]
  \centering
  \caption{桩的极限侧阻力标准值 $q_{sk}$ 和极限端阻力标准值 $q_{pk}$(单位:$\mathrm{kPa}$)}
  \label{tab:桩的极限指标}
  \begin{tabular}{ccccc}
      \toprule
      土层编号 & 土的名称 & 桩的侧阻力 $q_{sk}$ & 桩的端阻力 $q_{pk}$ & 抗拔系数 $\lambda$ \\
      \midrule
      \circled{1} & 素填土 & — & — & — \\
      \circled{2} & 淤泥质土 & — & — & — \\
      \circled{3} & 粉质粘土 & 27 & — & 0.71 \\
      \circled{4} & 粉质粘土 & 38 & 1800 & 0.75 \\
      \circled{5} & 粉质粘土混卵砾石 & 43 & 1800 & 0.75 \\
      \circled{6} & 强风化泥质粉砂岩 & 60 & 3500 & 0.65 \\
      \circled{7} & 中风化泥质粉砂岩 & — & — & 0.65 \\
      \bottomrule
  \end{tabular}
\end{table}

\subsection{水文地质条件}
\begin{enumerate}
  \item[1.] 拟建场区地下水对混凝土结构无腐蚀性。
  \item[2.] 地下水位深度:位于地表下3.5m。
\end{enumerate}

\subsection{场地条件}
建筑物所处场地抗震设防烈度为7度,场地内无可液化砂土,粉土。

\subsection{上部结构资料}

拟建建筑物为六层钢筋混凝土结构,长30m,宽9.6m。室外地坪标高同自然地面,室内外高差450mm。柱截面尺寸均为400mm×400mm,横向承重,柱网布置如图
\ref{fig:柱网布置图}所示。 
\\
\begin{figure}[htbp] % h:here, t:top, b:bottom, p:page
  \centering
  % 这里直接插入生成的 pdf 文件
  % width=0.9\textwidth 表示图片宽度占文本宽度的 90%,自动缩放
  \includegraphics[width=1\textwidth]{柱网布置图.pdf} 
  
  \caption{柱网布置图} % 图标题
  \label{fig:柱网布置图} % 用于引用的标签
\end{figure}

\newpage

\subsection{上部结构作用}

上部结构作用在柱底的荷载效应标准组合值如表\ref{tab:柱底荷载效应标准组合值}所示,该表中弯矩$M_k$ 、
水平力$V_k$均为横向方向。上部结构作用在柱底的荷载效应基本组合值如表\ref{tab:柱底荷载效应基本组合值}所示,
该表中弯矩$M$、水平力$V$均为横向方向。

\begin{table}[htbp]
  \centering
  \caption{柱底荷载效应标准组合值}
  \label{tab:柱底荷载效应标准组合值}
  \begin{tabular}{ccc}
      \toprule
      $F_k \, (\mathrm{kN})$ & $M_k\, (\mathrm{kN} \cdot \mathrm{m})$ & $V_k \, (\mathrm{kN})$ \\
      \midrule
      3728 & 277 & 200 \\
      \bottomrule
  \end{tabular}
\end{table}

\begin{table}[htbp]
  \centering
  \caption{柱底荷载效应基本组合值}
  \label{tab:柱底荷载效应基本组合值}
  \begin{tabular}{ccc}
      \toprule
      $F \, (\mathrm{kN})$ & $M\, (\mathrm{kN} \cdot \mathrm{m})$ & $V \, (\mathrm{kN})$ \\
      \midrule
      4641 & 331 & 235 \\
      \bottomrule
  \end{tabular}
\end{table}

\subsection{材料}
混凝土强度等级为C30,钢筋采用HRB400级。
