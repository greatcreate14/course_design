\section{桩的平面布置}

\subsection{确定桩数}  
根据《建筑桩基技术规范》(JGJ 94-2008) 第 5.1.1 条,单桩竖向力计算需考虑承台及覆土自重 $G_k$。因承台尺寸尚未确定,暂按上部结构荷载的 15\% 估算 $G_k$(即 $G_k = 0.15 F_k$),则初估桩数公式为:
$$n \ge \frac{1.15 F_k}{R_a} $$

则:
$$n \ge \frac{1.15 F_k}{R_a} = \frac{1.15 \times 3728}{781.2} = 5.49$$
桩取6根。

\subsection{桩的布置}
桩数为6根,选用矩形承台,桩排布为2行3列。

\subsection{承台尺寸}
根据《建筑桩基技术规范》(JGJ 94-2008) 第 3.3.3 条的表 3.3.3-1 、第 4.2.1 条、第 4.2.3 条和第 4.2.4 条规定: 


独立柱下桩基承台的最小宽度不应小于500mm,边桩中心至承台边缘的距离不应小于桩的直径或边长,且桩的外边缘至承台边缘的距离不应小于150mm。对于墙下条形承台梁,桩的外边缘至承台梁边缘的距离不应小于75mm。承台的最小厚度不应小于300mm。 


承台底面钢筋的混凝土保护层厚度,当有混凝土垫层时,不应小于 50mm,无垫层时不应小于70mm;此外尚不应小于桩头嵌入承台内的长度。 


桩嵌入承台内的长度对中等直径桩不宜小于50mm;对大直径桩不宜小于100mm。
\\
本设计桩中心距:$$s = 4d = 4 \times 0.45 = 1.8 \text{ m}$$
边桩中心至承台边缘的距离取0.45m,承台底面钢筋的混凝土保护层厚度取70mm,桩顶嵌入承台长度取100mm,设承台高$h = 1.5 \text{m}$,
\\则承台的长边长:$$ a = 2 \times 0.45 + 2 \times 1.8 = 4.5 \text{ m}$$
承台短边长: $$
    b  = 2 \times 0.45 + 1.8 = 2.7 \text{ m}
$$
承台有效高度: $$
    h_0 = 1.5-0.07 = 1.43 \text{ m}
$$
