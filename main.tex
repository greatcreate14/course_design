%!TEX program = xelatex
\documentclass[a4paper, twoside]{ctexart} 

% ==================== 导言区设置 ====================
\usepackage[top=2.5cm, bottom=2.5cm, inner=3cm, outer=2.5cm]{geometry} 
\usepackage{fontspec} 
\usepackage{xeCJK} 
\usepackage{amsmath, amssymb, amsthm} 
\usepackage{graphicx} 
\usepackage{booktabs} 
\usepackage{multirow} 
\usepackage{indentfirst} 
\usepackage{listings} 
\usepackage{xcolor} 
\usepackage{fancyhdr}
\usepackage{pdfpages} % <--- 新增:用于插入PDF封面
\usepackage{enumitem} % 用于自定义列表样式
\usepackage{pifont}
\usepackage{booktabs} % 用于绘制专业的三线表
\usepackage{array}    % 更好的列控制

% 参考文献使用 BibTeX + natbib 宏包
% 顺序编码制
\usepackage[numbers, sort&compress]{natbib}
\bibliographystyle{plain}

\usepackage{hyperref} %超链接

\makeatletter
\def\input@path{{scr/}} % 这里的 scr/ 是你的文件夹名称
\makeatother

\graphicspath{{figures/}}

% 设置超链接颜色
\hypersetup{
    colorlinks=true,
    linkcolor=black,
    citecolor=black
}

\setlength{\headheight}{12.64723pt}

% 设置章节标题格式
\ctexset{
  section={
    format=\Large\bfseries\centering,
    nameformat=\bfseries,
    number=\chinese{section},
    aftername={、},
    beforeskip=1.5ex plus 0.2ex minus 0.2ex,
    afterskip=1ex plus 0.1ex
  },
  subsection={
    format=\large\bfseries,
    nameformat=\bfseries,
    number=\arabic{section}.\arabic{subsection},
    aftername={\quad},
    beforeskip=1.2ex plus 0.1ex minus 0.1ex,
    afterskip=0.8ex plus 0.1ex
  },
  subsubsection={
    number=\arabic{subsubsection},
    aftername={. }
  }
}

% 定理环境设置
\newtheorem{theorem}{定理}[section]
\newtheorem{lemma}[theorem]{引理}
\newtheorem{definition}[theorem]{定义}
\newtheorem{example}[theorem]{例}

% 代码样式设置
\lstset{
  basicstyle=\small\ttfamily,
  keywordstyle=\bfseries, 
  commentstyle=\itshape, 
  frame=none, 
  numbers=none 
}

% 页眉页脚设置
\pagestyle{fancy}
\fancyhf{}
% RO: 奇数页右侧,LE: 偶数页左侧(即外侧打印页码)
\fancyhead[RO, LE]{\thepage}
% LO: 奇数页左侧,RE: 偶数页右侧(即内侧打印章节名)
\fancyhead[LO, RE]{\leftmark}

\renewcommand{\headrulewidth}{0.4pt}
\fancyfoot[C]{}

% 使自动插入的空白页不显示页眉页脚
\makeatletter
\def\cleardoublepage{\clearpage\if@twoside \ifodd\c@page\else
  \hbox{}\thispagestyle{empty}\newpage\if@twocolumn\hbox{}\newpage\fi\fi\fi}
\makeatother

% --- 图表按章节编号设置 ---
% 使用 amsmath 提供的命令将图表编号与 section 绑定
\numberwithin{figure}{section}
\numberwithin{table}{section}

% 如果你希望即使 section 使用中文数字(如“第一页”),图表依然显示阿拉伯数字(如“图1.1”)
% 请取消下面两行的注释并使用它们:
 \renewcommand{\thefigure}{\arabic{section}.\arabic{figure}}
 \renewcommand{\thetable}{\arabic{section}.\arabic{table}}

% 自定义命令
\newcommand{\R}{\mathbb{R}} 
\newcommand{\diff}{\mathop{}\!\mathrm{d}} 
\newcommand{\circled}[1]{\ding{\numexpr 171 + #1 \relax}}

% ==================== 正文区域 ====================
\begin{document}

% 1. 插入封面 PDF (假设文件名为 cover.pdf)
% pages=1 表示只取第一页,如果是多页封面可以用 pages=-
\includepdf[pages=1]{课设说明.pdf} 
\cleardoublepage %封面单独打印
\includepdf[pages=2]{课设说明.pdf} 
\cleardoublepage  %课设任务页单独打印

% 2. 生成目录
\pagenumbering{gobble}
\tableofcontents
\cleardoublepage %确保正文从奇数页开始

% 3. 导入各个部分

\pagenumbering{arabic}

% 导入各章节
\section{设计资料}

\subsection{地形}
拟建建筑场地地势平坦,局部堆有建筑垃圾。

\subsection{工程地质条件}
自上而下土层依次如表\ref{tab:场地土层物理力学指标统计表}所示:
\begin{table}[htbp]
  \centering
  \caption{场地土层物理力学指标统计表}
  \label{tab:场地土层物理力学指标统计表}
  \begin{tabular}{clclc}
      \toprule
      层号 & 土层名称 & 层厚 (m) & \multicolumn{1}{c}{状态描述} & $f_{ak}$ (kPa) \\
      \midrule
      \circled{1} & 杂填土 & 2.0 & 稍湿,松散 & 95 \\
      \circled{2} & 淤泥质土 & 3.5 & 流塑 & 65 \\
      \circled{3} & 粉质粘土 & 5.0 & 稍密 & 130 \\
      \circled{4} & 粉质粘土 & 10.0 & 湿,可塑 & 200 \\
      \circled{5} & 粉质粘土混卵砾石 & 4.2 & - & 220 \\
      \circled{6} & 强风化泥质粉砂岩 & 10.0 & - & 300 \\
      \circled{7} & 中风化泥质粉砂岩 & 未揭穿 & - & 1000 \\
      \bottomrule
  \end{tabular}
\end{table}


\subsection{岩土设计技术参数}
岩土设计参数如表\ref{tab:地基岩土物理力学参数}和表\ref{tab:桩的极限指标}所示:

\begin{table}[htbp]
  \centering
  \caption{地基岩土物理力学参数}
  \label{tab:地基岩土物理力学参数}
  % 定义列格式:居中对齐,部分列宽微调以适应页面
  \begin{tabular}{cccccccc}
      \toprule
      土层 & 土层 & 孔隙比 & 含水量 & 液性 & $\gamma$ & $C(\text{kPa})/$ & 压缩模量 \\
      编号 & 名称 & $e$ & $W(\%)$ & 指数 $I_L$ & $(\text{kN}/\text{m}^3)$ & $\Phi(^\circ)$ & $E_s(\text{MPa})$ \\
      \midrule
      \circled{1}& 杂填土 & — & — & — & 19 & 20/10 & 5.0 \\
      \circled{2} & 淤泥质土 & 1.04 & 62.4 & 1.08 & 17 & 8/5 & 3.8 \\
      \circled{3} & 粉质粘土 & — & — & — & 20 & 38/16 & 5.81 \\
      \circled{4} & 粉质黏土 & — & — & — & 19.7 & 42/17.4 & 8.18 \\
      \circled{5} & \begin{tabular}{@{}c@{}}粉质粘土\\混卵砾石\end{tabular} & — & — & — & 20 & 50/20 & 9.0 \\
      \circled{6} & \begin{tabular}{@{}c@{}}强风化泥\\质粉砂岩\end{tabular} & — & — & — & 20.5 & 20/35 & 15 \\
      \circled{7} & \begin{tabular}{@{}c@{}}中风化泥\\质粉砂岩\end{tabular} & — & — & — & 21.5 & 200/40 & — \\
      \bottomrule
  \end{tabular}
\end{table}
\newpage

\begin{table}[htbp]
  \centering
  \caption{桩的极限侧阻力标准值 $q_{sk}$ 和极限端阻力标准值 $q_{pk}$(单位:$\mathrm{kPa}$)}
  \label{tab:桩的极限指标}
  \begin{tabular}{ccccc}
      \toprule
      土层编号 & 土的名称 & 桩的侧阻力 $q_{sk}$ & 桩的端阻力 $q_{pk}$ & 抗拔系数 $\lambda$ \\
      \midrule
      \circled{1} & 素填土 & — & — & — \\
      \circled{2} & 淤泥质土 & — & — & — \\
      \circled{3} & 粉质粘土 & 27 & — & 0.71 \\
      \circled{4} & 粉质粘土 & 38 & 1800 & 0.75 \\
      \circled{5} & 粉质粘土混卵砾石 & 43 & 1800 & 0.75 \\
      \circled{6} & 强风化泥质粉砂岩 & 60 & 3500 & 0.65 \\
      \circled{7} & 中风化泥质粉砂岩 & — & — & 0.65 \\
      \bottomrule
  \end{tabular}
\end{table}

\subsection{水文地质条件}
\begin{enumerate}
  \item[1.] 拟建场区地下水对混凝土结构无腐蚀性。
  \item[2.] 地下水位深度:位于地表下3.5m。
\end{enumerate}

\subsection{场地条件}
建筑物所处场地抗震设防烈度为7度,场地内无可液化砂土,粉土。

\subsection{上部结构资料}

拟建建筑物为六层钢筋混凝土结构,长30m,宽9.6m。室外地坪标高同自然地面,室内外高差450mm。柱截面尺寸均为400mm×400mm,横向承重,柱网布置如图
\ref{fig:柱网布置图}所示。 
\\
\begin{figure}[htbp] % h:here, t:top, b:bottom, p:page
  \centering
  % 这里直接插入生成的 pdf 文件
  % width=0.9\textwidth 表示图片宽度占文本宽度的 90%,自动缩放
  \includegraphics[width=1\textwidth]{柱网布置图.pdf} 
  
  \caption{柱网布置图} % 图标题
  \label{fig:柱网布置图} % 用于引用的标签
\end{figure}

\newpage

\subsection{上部结构作用}

上部结构作用在柱底的荷载效应标准组合值如表\ref{tab:柱底荷载效应标准组合值}所示,该表中弯矩$M_k$ 、
水平力$V_k$均为横向方向。上部结构作用在柱底的荷载效应基本组合值如表\ref{tab:柱底荷载效应基本组合值}所示,
该表中弯矩$M$、水平力$V$均为横向方向。

\begin{table}[htbp]
  \centering
  \caption{柱底荷载效应标准组合值}
  \label{tab:柱底荷载效应标准组合值}
  \begin{tabular}{ccc}
      \toprule
      $F_k \, (\mathrm{kN})$ & $M_k\, (\mathrm{kN} \cdot \mathrm{m})$ & $V_k \, (\mathrm{kN})$ \\
      \midrule
      3728 & 277 & 200 \\
      \bottomrule
  \end{tabular}
\end{table}

\begin{table}[htbp]
  \centering
  \caption{柱底荷载效应基本组合值}
  \label{tab:柱底荷载效应基本组合值}
  \begin{tabular}{ccc}
      \toprule
      $F \, (\mathrm{kN})$ & $M\, (\mathrm{kN} \cdot \mathrm{m})$ & $V \, (\mathrm{kN})$ \\
      \midrule
      4641 & 331 & 235 \\
      \bottomrule
  \end{tabular}
\end{table}

\subsection{材料}
混凝土强度等级为C30,钢筋采用HRB400级。
 
\newpage
\section{桩基设计}

\subsection{桩基持力层和桩长}

根据《建筑桩基技术规范》(JGJ 94-2008) 第 3.3.3 条规定:应选择较硬土层作为桩端持力层。桩端全断面进入持力层的深度,对于黏性土、粉土不宜小于 $2d$,砂土不宜小于 $1.5d$,碎石类土不宜小于 $1d$。

本设计选用第\circled{5}层粉质粘土混卵砾石作为桩端持力层。
桩截面边长 $d = 500\text{mm}$。
设计桩端进入持力层深度为:
\begin{equation*}
    3.0d =  3 \times 0.5=1.5 \text{m}
\end{equation*}

设地面标高为-0.45m,初步设计承台高1.0m,承台地面埋置深度标高-1.45m,根据《建筑桩基技术规范》(JGJ 94-2008) 第 4.2.4 条规定:桩嵌入承台内的长度对中等直径桩不宜小于50mm;对大直径桩不宜小于100mm。本设计桩顶伸入承台100mm。


由表\ref{tab:地基岩土物理力学参数}场地地质条件可知各个土层的厚度,故桩基有效桩长$l$为:
\begin{equation*}
    l = 2+3.5+5+10+1.5-1 =21 \text{m}
\end{equation*}

\subsection{单桩竖向承载力的确定}

由表\ref{tab:桩的极限指标}可知桩的极限侧阻力标准值 $q_{sk}$ 和极限端阻力标准值 $q_{pk}$
根据规范经验公式,其单桩竖向承载力计算公式为:
\begin{equation*}
    Q_{uk}=Q_{sk}+Q_{pk}=u\Sigma q_{sik} l_i + q_{pk} A_p
\end{equation*}
桩身周长:
\begin{equation*}
    u=4\times 0.5 = 2 \text{m}
\end{equation*}
桩截面积:
\begin{equation*}
    A_p = 0.5 \times 0.5 = 0.25 \text{m}^2
\end{equation*}
则:
\begin{equation*}
    \begin{aligned}
        Q_{sk}  &= u\Sigma q_{sik} l_i = 2.0 \times (27 \times 5 + 38 \times 10 + 43 \times 1.5) = 1159 \text{ kN}\\
    Q_{pk} &= q_{pk} A_p = 1800 \times 0.25 = 450 \text{ kN} \\
    Q_{uk} &= Q_{sk} + Q_{pk} = 1159 + 450 = 1609 \text{ kN}
    \end{aligned}
\end{equation*}
\\
不考虑群桩效应,估算单桩竖向承载力设计值 $R_a$ 为:
\begin{equation*}
    R_a = \frac{Q_{uk}}{K} = \frac{1609}{2} =804.5 \text{ kN}
\end{equation*}

\subsection{确定桩数}  
根据《建筑桩基技术规范》(JGJ 94-2008) 第 5.1.1 条,单桩竖向力计算需考虑承台及覆土自重 $G_k$。因承台尺寸尚未确定,暂按上部结构荷载的 15\% 估算 $G_k$(即 $G_k = 0.15 F_k$),则初估桩数公式为:
$$n \ge \frac{1.15 F_k}{R_a} $$

则:
$$n \ge \frac{1.15 F_k}{R_a} = \frac{1.15 \times 3728}{804.5} = 5.3$$
桩取6根。 
\newpage

% 参考文献
\addcontentsline{toc}{section}{参考文献} % 将参考文献加入目录
\nocite{*} % .bib 文件里的所有条目都会出现在列表里
\bibliography{ref/refs}  % 参考文献使用 BibTeX 编译

%最后一章不用\newpage
\cleardoublepage %确保成绩评定表从奇数页开始
\includepdf[pages=3]{课设说明.pdf} 

\end{document}